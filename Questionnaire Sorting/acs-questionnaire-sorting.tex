\documentclass[
ngerman,
accentcolor=9c,% Farbe für Hervorhebungen auf Basis der Deklarationen in den
identbarcolor=9c,
]{tudaexercise}

\usepackage[main=english, ngerman]{babel}
\usepackage{tikz}
\usetikzlibrary{patterns}
\usetikzlibrary{plotmarks}
\usetikzlibrary{arrows}
\usepackage{vwcol}

\newcommand{\probBox}{
	\begin{tikzpicture}[scale=1]
		% Variables
		\def\boxHeight{5} % in cm
		\def\boxWidth{5} % in cm
		
		\def\xlabel{Prozent korrekt}
		
		% Box
		\node[rectangle, draw=black, rounded corners=0mm, minimum width=\boxWidth cm, minimum height=\boxHeight cm, line width=1pt] at (\boxWidth/2,\boxHeight/2) {};
		
		% x-axis tiks
		\def\ytop{0}
		\def\ybottom{-0.1}
		\draw[line width=1pt] (0, \ytop) -- (0, \ybottom) node[anchor=north] {\large $0$};
		\draw[line width=1pt] (0.2*\boxWidth, \ytop) -- (0.2*\boxWidth, \ybottom) node[anchor=north] {\large $20$};
		\draw[line width=1pt] (0.4*\boxWidth, \ytop) -- (0.4*\boxWidth, \ybottom) node[anchor=north] {\large $40$};
		\draw[line width=1pt] (0.6*\boxWidth, \ytop) -- (0.6*\boxWidth, \ybottom) node[anchor=north] {\large $60$};
		\draw[line width=1pt] (0.8*\boxWidth, \ytop) -- (0.8*\boxWidth, \ybottom) node[anchor=north] {\large $80$};
		\draw[line width=1pt] (1*\boxWidth, \ytop) -- (1*\boxWidth, \ybottom) node[anchor=north] {\large $100$};
		
		% y-axis tiks
		\def\xright{0}
		\def\xleft{-0.1}
		\draw[line width=1pt] (\xright, 0) -- (\xleft, 0) node[anchor=east] {\large $0$};
		\draw[line width=1pt] (\xright, 0.2*\boxHeight) -- (\xleft, 0.2*\boxHeight) node[anchor=east] {\large $0.2$};
		\draw[line width=1pt] (\xright, 0.4*\boxHeight) -- (\xleft, 0.4*\boxHeight) node[anchor=east] {\large $0.4$};
		\draw[line width=1pt] (\xright, 0.6*\boxHeight) -- (\xleft, 0.6*\boxHeight) node[anchor=east] {\large $0.6$};
		\draw[line width=1pt] (\xright, 0.8*\boxHeight) -- (\xleft, 0.8*\boxHeight) node[anchor=east] {\large $0.8$};
		\draw[line width=1pt] (\xright, 1*\boxHeight) -- (\xleft, 1*\boxHeight) node[anchor=east] {\large $1$};	
		
		% Draw grid lines
		% \draw[step=.5cm, color=gray] (0,0) grid (\boxWidth, \boxHeight);
		
		% Captions
		\node[] at (\boxWidth/2, -.8) {\xlabel}; % x-axis
		\node[rotate=90] at (-1.1, \boxHeight/2) {Wahrscheinlichkeit}; % y-axis
	\end{tikzpicture}
}


\newcommand{\sortingBoxes}{
	\begin{tikzpicture}
		\foreach \i in {1, 2, 3, 4, 5, 6, 7, 8, 9, 10} {
			\def\yPos{-\i*.6}
			% box
			\node[shape=rectangle, draw=black, minimum width=4.5cm, minimum height=.5cm, line width=.5pt, label={[shift={(-2.5,-.5)}]\i.}] 
			at (0,\yPos) {};
			\node[shape=rectangle, minimum height=5mm, minimum width=5mm, draw=black, line width=.5pt] at (2.8,\yPos) {};
		}
	\end{tikzpicture}
}

\newcommand{\myTask}[2]{
	\begin{task}[]{#1}
		\begin{minipage}{.62\textwidth}
			#2
		\end{minipage}
		\begin{minipage}{.38\textwidth}
			\probBox
		\end{minipage}
	\vspace{-1mm}
	\end{task}
}

\newcommand{\myTaskEmptyBox}[1]{
	\myTask{#1}{
		\vspace{-0.8cm}
		\fbox{\color{white}\rule{\textwidth - 1cm}{4.8cm}\color{black}}
	}
}

%%%%%%%%%%%%%%%%%%%%%%%%%

\title{Predictions on Task Performance\\- Sorting -}



\begin{document}
\maketitle
\vspace{-3mm}

%%%%%% GENERAL INFORMATION %%%%%%

\textbf{Assignment:}
\begin{itemize}
	\item Please answer each question in the questionnaire as good as possible.
	\item Directly afterwards, draw a probability density function inside the given grid.
	\item It is possible to achieve half points with a maximum of 5 points.
\end{itemize}

%%%%%% TASK 1 %%%%%%
	
\begin{task}[]{Sortieren Sie die Städte nach ihrem Längengrad (1 = nördlichste Stadt, 10 = südlichste Stadt)}
	\begin{minipage}{.29\textwidth}
		\vspace{-6mm}
		\textbf{Städte:}
		\begin{itemize}
			\itemsep0em
			\item Kiel
			\item Hamburg
			\item Bremen
			\item Berlin
			\item Hannover
			\item Rotterdam
			\item London
			\item Köln
			\item FFM
			\item Nürnberg
		\end{itemize}
	\end{minipage}
	\begin{minipage}{.35\textwidth}
		\vspace{-4mm}
		\sortingBoxes
	\end{minipage}
	\begin{minipage}{.36\textwidth}
		\probBox
	\end{minipage}
	\vspace{-1mm}
\end{task}

%%%%%% TASK 2 %%%%%%

\begin{task}[]{Sortieren Sie die Kanzler nach ihrer Amtszeit (1 = am frühsten, 10 = am spätesten)}
	\begin{minipage}{.29\textwidth}
		\vspace{-6mm}
		\textbf{Kanzler:}
		\begin{itemize}
			\itemsep0em
			\item Erhard
			\item Adenauer
			\item Scheel
			\item Kohl
			\item Brandt
			\item Schröder
			\item Schmidt
			\item Merkel
			\item Bismarck
			\item Kiesinger
		\end{itemize}
	\end{minipage}
	\begin{minipage}{.35\textwidth}
		\vspace{-4mm}
		\sortingBoxes
	\end{minipage}
	\begin{minipage}{.36\textwidth}
		\probBox
	\end{minipage}
	\vspace{-1mm}
\end{task}

%%%%% TASK 3 %%%%%

\begin{task}[]{Sortieren Sie die Todesursachen nach ihrer Häufigkeit in Deutschland in 2017\\(1 = am häufigsten, 10 = am seltensten)}
	\begin{minipage}{.29\textwidth}
		\vspace{-2mm}
		\textbf{Krankheiten:}
		\begin{itemize}
			\itemsep0em
			\item Herzinsuffizienz
			\item Demenz
			\item Lungen- und Bronchialkrebs
			\item Pneumonie
			\item Herzinfarkt
			\item Vorhofflimmern \& Vorhofflattern
			\item Chronische ischämische Herzkrankheit
			\item Brustdrüsenkrebs
			\item Hypertensive Herzkrankheit
			\item Sonstige chronische obstruktive Lungenkrankheit
		\end{itemize}
	\end{minipage}
	\begin{minipage}{.35\textwidth}
		\vspace{-4mm}
		\sortingBoxes
	\end{minipage}
	\begin{minipage}{.36\textwidth}
		\probBox
	\end{minipage}
	\vspace{-1mm}
\end{task}

%%%%%% TASK 4 %%%%%

\begin{task}[]{Sortieren Sie die Berufe nach ihrem Durchschnittseinkommen \\(Mittleres Bruttogehalt pro Monat in Deutschland; 1 = am höchsten, 10 = am niedrigsten)}
	\begin{minipage}{.29\textwidth}
		\vspace{-6mm}
		\textbf{Berufe:}
		\begin{itemize}
			\itemsep0em
			\item Koch/Köchin
			\item Informatiker/in
			\item Bürokaufmann/-frau
			\item Friseur/-in
			\item Psychologe/-in
			\item Bankkaufmann/-frau
			\item Altenpfleger/-in
			\item Fahrlehrer/-in
			\item Arzt/Ärztin
			\item Lehrer/in Gymnasium
		\end{itemize}
	\end{minipage}
	\begin{minipage}{.35\textwidth}
		\vspace{-4mm}
		\sortingBoxes
	\end{minipage}
	\begin{minipage}{.36\textwidth}
		\probBox
	\end{minipage}
	\vspace{-1mm}
\end{task}

%%%%%% TASK 5 %%%%%

\begin{task}[]{Sortieren Sie die ``Persönlichkeiten'' nach ihrer Follower-Zahl auf Instagram \\(1 = am meisten, 10 = am wenigsten)}
	\begin{minipage}{.29\textwidth}
		\vspace{-6mm}
		\textbf{Persönlichkeiten:}
		\begin{itemize}
			\itemsep0em
			\item Selena Gomez
			\item Kim Kardashian
			\item Neymar Jr.
			\item Ariana Grande
			\item Beyonce
			\item The Rock
			\item Kylie Jenner
			\item Christiano Ronaldo
			\item Instagram
			\item Leonel Messi
		\end{itemize}
	\end{minipage}
	\begin{minipage}{.35\textwidth}
		\vspace{-4mm}
		\sortingBoxes
	\end{minipage}
	\begin{minipage}{.36\textwidth}
		\probBox
	\end{minipage}
	\vspace{-1mm}
\end{task}

	





%\myTask{Farbkreis}{
%		Schreiben Sie die Namen der korrekten Farben in den Farbkreis.
%}


\end{document}