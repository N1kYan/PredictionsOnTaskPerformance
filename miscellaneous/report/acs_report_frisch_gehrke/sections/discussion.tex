\documentclass[../main/main.tex]{subfiles}

% Put everything that shall appear in the discussion
% inside this document environment.
\begin{document}
	\textbf{What to do?}
	\begin{itemize}
		\item discuss challenges that you faced during implementation, 
		\item reflect your solution
		\item give an outlook
	\end{itemize}

	\noindent\rule{\textwidth}{.4pt}
	
	\vspace{2mm}
	
	\noindent Our universal goal was to design an experiment which can understand how well people estimate their performance. Therefore is our study a preliminary study. For this we designed an experiment, where subjects had to solve several five item sorting tasks. After finishing each task, the subject had to draw a probability density function over their performance. The questionnaire had two conditions. The first condition was sorting without active recall. The five answers to the question were already given in a randomized order and had to be sorted in the write order. The second condition was sorting with active recall. The question was open and the subject had to know the items and the correct ordering. In total we had 7 questions for condition one and one question for condition two.
	
	After collecting the answers from XX subjects, our program read the probability density functions with the help of computer vision and we analyzed the the Brier score of the data.
	
	\textbf{Objectivität Finding the correct type of questions.}
	
	\textbf{Finding the correct questions.}
	
	\textbf{Finding the correct metric.}
	
	\textbf{Explaining the concept of probability density functions.}
	
	\textbf{Computer Vision.}
	
\end{document}