\documentclass[
ngerman,
accentcolor=9c,% Farbe für Hervorhebungen auf Basis der Deklarationen in den
identbarcolor=9c,
]{tudaexercise}

\usepackage[main=english, ngerman]{babel}
\usepackage{tikz}
\usetikzlibrary{patterns}
\usetikzlibrary{plotmarks}
\usetikzlibrary{arrows}
\usepackage{vwcol}
\usepackage{graphicx}
\usepackage{grffile}

\newcommand{\probBox}{
	\begin{tikzpicture}[scale=1]
		% Variables
		\def\boxHeight{4.5} % in cm
		\def\boxWidth{4.5} % in cm
		
		\def\xlabel{Prozent korrekt}
		
		% Box
		\node[rectangle, draw=black, rounded corners=0mm, minimum width=\boxWidth cm, minimum height=\boxHeight cm, line width=1pt] at (\boxWidth/2,\boxHeight/2) {};
		
		% x-axis tiks
		\def\ytop{0}
		\def\ybottom{-0.1}
		\draw[line width=1pt] (0, \ytop) -- (0, \ybottom) node[anchor=north] {\large $0$};
		\draw[line width=1pt] (0.2*\boxWidth, \ytop) -- (0.2*\boxWidth, \ybottom) node[anchor=north] {\large $20$};
		\draw[line width=1pt] (0.4*\boxWidth, \ytop) -- (0.4*\boxWidth, \ybottom) node[anchor=north] {\large $40$};
		\draw[line width=1pt] (0.6*\boxWidth, \ytop) -- (0.6*\boxWidth, \ybottom) node[anchor=north] {\large $60$};
		\draw[line width=1pt] (0.8*\boxWidth, \ytop) -- (0.8*\boxWidth, \ybottom) node[anchor=north] {\large $80$};
		\draw[line width=1pt] (1*\boxWidth, \ytop) -- (1*\boxWidth, \ybottom) node[anchor=north] {\large $100$};
		
		% y-axis tiks
		\def\xright{0}
		\def\xleft{-0.1}
		\draw[line width=1pt] (\xright, 0) -- (\xleft, 0) node[anchor=east] {\large $0$};
		\draw[line width=1pt] (\xright, 0.2*\boxHeight) -- (\xleft, 0.2*\boxHeight) node[anchor=east] {\large $0.2$};
		\draw[line width=1pt] (\xright, 0.4*\boxHeight) -- (\xleft, 0.4*\boxHeight) node[anchor=east] {\large $0.4$};
		\draw[line width=1pt] (\xright, 0.6*\boxHeight) -- (\xleft, 0.6*\boxHeight) node[anchor=east] {\large $0.6$};
		\draw[line width=1pt] (\xright, 0.8*\boxHeight) -- (\xleft, 0.8*\boxHeight) node[anchor=east] {\large $0.8$};
		\draw[line width=1pt] (\xright, 1*\boxHeight) -- (\xleft, 1*\boxHeight) node[anchor=east] {\large $1$};	
		
		% Draw grid lines
		% \draw[step=.5cm, color=gray] (0,0) grid (\boxWidth, \boxHeight);
		
		% Captions
		\node[] at (\boxWidth/2, -.8) {\xlabel}; % x-axis
		\node[rotate=90] at (-1.1, \boxHeight/2) {Wahrscheinlichkeit}; % y-axis
	\end{tikzpicture}
}


\newcommand{\sortingBoxes}{
	\begin{tikzpicture}
		\foreach \i in {1, 2, 3, 4, 5} {
			\def\yPos{-\i*.8}
			% box
			\node[shape=rectangle, draw=black, minimum width=4.5cm, minimum height=.7cm, line width=.5pt, label={[shift={(-2.5,-.6)}]\i.}] 
			at (0,\yPos) {};
			
			\node[shape=rectangle, draw=black, minimum width=7mm, minimum height=7mm, line width=.5pt] 
			at (3.4,\yPos) {};
			
			\draw (2.7,-0.45) -- (2.7, -4.35);
			
			
		}
	\end{tikzpicture}
}

\newcommand{\evalBoxes}{
	\begin{tikzpicture}[scale=1]
		\node[shape=rectangle, draw=black, minimum height=4cm, minimum width=7mm] at (14, 8) {};
		
	\end{tikzpicture}
}

\newcommand{\myTask}[2]{
	\begin{task}[]{#1}
		\begin{minipage}{.62\textwidth}
			#2
		\end{minipage}
		\begin{minipage}{.38\textwidth}
			\probBox
		\end{minipage}
	\vspace{-1mm}
	\end{task}
}

\newcommand{\myTaskEmptyBox}[1]{
	\myTask{#1}{
		\vspace{-0.8cm}
		\fbox{\color{white}\rule{\textwidth - 1cm}{4.8cm}\color{black}}
	}
}

% TEMPLATE for the sorting task
\newcommand{\sortingTask}[2] {
	\begin{task}[]{#1}
		\begin{minipage}{.25\textwidth}
			\vspace{-6mm}
			#2
			\hfill
		\end{minipage}
		\begin{minipage}{.4\textwidth}
			\vspace{-4mm}
			\sortingBoxes
		\end{minipage}
		\begin{minipage}{.35\textwidth}
			\probBox
		\end{minipage}
		\vspace{-1mm}
	\end{task}
}

%%%%%%%%%%%%%%%%%%%%%%%%%

\title{Predictions on Task Performance\\- Sorting -}



\begin{document}
\maketitle
\vspace{-3mm}

%%%%%% GENERAL INFORMATION %%%%%%

\textbf{Versuchspersonen-Code}
\begin{itemize}
	\begin{minipage}{.45\textwidth}
		\vspace{-2cm}
		\itemsep.55em
		\item Zweiter Buchstabe des Vornamens des Vaters
		\item Vorletzter Buchstabe des Vornamens der Mutter
		\item Letzter Buchstabe ihres Nachnamens
		\item Vorletzter Buchstabe Ihres Vornamens	
	\end{minipage}
		\begin{tikzpicture}
			\foreach \i in {1, 2, 3, 4} {
				\def\yPos{-2 -\i*.6}
				\node[shape=rectangle, draw=black, minimum width=5mm, minimum height=5mm, line width=.5pt] 
				at (0,\yPos) {};
			}
		\end{tikzpicture}
	\begin{minipage}{.55\textwidth}
		
	\end{minipage}
\end{itemize}

\textbf{Aufgabe:}
\begin{itemize}
	\item Im Folgenden werden Sie dazu aufgefordert Begriffe in eine bestimmte Reihenfolge zu sortieren.
	\item Schreiben Sie dafür die Begriffe in die rechteckigen Kästchen in der Mitte jeder Aufgabe.\\Bitte lassen Sie die kleinen Quadrate auf alle Fälle frei!
	\item Nachdem Sie die Begriffe einer Aufgabe sortiert haben, zeichnen Sie bitte die Wahrscheinlichkeitsverteilung Ihrer erwarteten Leistung in das  Quadrat auf der rechten Seite.
	
\end{itemize}

\textbf{Wahrscheinlichkeitsverteilungen:}
\begin{itemize}
	\item Eine Wahrscheinlichkeitsverteilung ordnet jedem Eintrag auf der x-Achse einen Wahrscheinlichkeitswert (zwischen 0 und 1) auf der y-Achse zu.
	\item Dadurch wird jedem möglichen Ereignis die Wahrscheinlichkeit zugewiesen, mit der es eintreffen kann.
	\item Ihre Antworten in den folgenden Aufgaben werden bewertet. Sie sollen für jede eine Einschätzung einzeichnen, für wie wahrscheinlich Sie es halten, die Aufgabe zu 0\%, 10\%, 30\%, 50\%, usw.	 korrekt gelöst zu haben. Auf der x-Achse befinden sich also die prozentualen Anteile der maximal zu erreichenden Punktzahl. 
	\item Im Folgenden finden Sie dazu einige Beispiele.
\end{itemize}
\begin{center}
	\begin{minipage}{.235\textwidth}
		\includegraphics[width=\linewidth]{{example_distribution_5}.png}
	\end{minipage}
	\begin{minipage}{.235\textwidth}
		\includegraphics[width=\linewidth]{{example_distribution_4}.png}
	\end{minipage}
	\begin{minipage}{.235\textwidth}
		\includegraphics[width=\linewidth]{{example_distribution_1}.png}
	\end{minipage}
	\begin{minipage}{.235\textwidth}
		\includegraphics[width=\linewidth]{{example_distribution_2}.png}
	\end{minipage}
\end{center}
\begin{itemize}
	\item Das Beispiel links außen zeigt eine Verteilung, welche repräsentiert, dass Sie vermutlich ca. $\frac{2}{3}$ der maximal zu erreichenden Punkte erhalten werden, Sie wollen aber auch nicht ausschließen, dass es etwas mehr oder weniger Punkte sein können. Diese Unsicherheit ist im zweiten Bild deutlich geringer. Sie sind sich hier relativ sicher, dass sie ca. $\frac{1}{3}$ der möglichen Punkte für diese Aufgabe erhalten werden. Im dritten Bild sehen Sie eine Verteilung, die ausdrückt, dass Sie ihre Leistung bei der aktuellen Aufgabe als relativ schlecht beurteilen. Das Bild rechts außen zeigt eine Verteilung, welche impliziert, dass Sie nichts über ihre erreichte Punktzahl sagen können und jede Leistung gleich wahrscheinlich ist.
	\item Bitte achten Sie darauf, den Graphen möglichst \textit{durchgehend }vom linken zum rechten Rand des jeweiligen Kastens einzuzeichnen.
\end{itemize}
\newpage

%%%%%% TASK 1 %%%%%%
	
\sortingTask{
	Sortieren Sie die Städte nach ihrem Längengrad (1 = nördlichste Stadt, 10 = südlichste Stadt).
}{
	\textbf{Städte:}
	\begin{itemize}
		\itemsep0em
		\item Hamburg
		\item Kiel
		\item Hannover
		\item Bremen
		\item Göttingen
	\end{itemize}
}

%%%%%% TASK 2 %%%%%%

\sortingTask{
	Sortieren Sie die Kanzler nach ihrer Amtszeit (1 = am frühsten, 5 = am spätesten).
}{
	\textbf{Kanzler:}
	\begin{itemize}
		\itemsep0em
		\item Adenauer
		\item Kohl
		\item Kiesinger
		\item Brandt
		\item Schmidt
	\end{itemize}
}

%%%%% TASK 3 %%%%%

\sortingTask{
	Sortieren Sie die Todesursachen für Frauen in Deutschland in 2017 nach ihrer Häufigkeit\\(1 = am häufigsten, 5 = am seltensten).
}{
	\textbf{Krankheiten:}
	\begin{itemize}
		\itemsep0em
		\item Lungenentzündung
		\item Herzversagen
		\item Demenz
		\item Herzinfarkt
		\item Brustdrüsenkrebs
	\end{itemize}
}

%%%%%% TASK 4 %%%%%

\sortingTask{
	Sortieren Sie die Berufe nach ihrem Durchschnittseinkommen \\(Mittleres Bruttogehalt pro Monat in Deutschland; 1 = am höchsten, 5 = am niedrigsten).
}{
	\textbf{Berufe:}
	\begin{itemize}
		\itemsep0em
		\item Altenpfleger/-in
		\item Psychologe/-in
		\item Informatiker/-in
		\item Arzt / Ärztin
		\item Koch / Köchin
	\end{itemize}
}

%%%%%% TASK 5 %%%%%

\sortingTask{
	Sortieren Sie die Sprachen nach der Anzahl ihrer Muttersprachler (1 = am meisten, 5 = am wenigsten).
}{
	\textbf{Sprachen:}
	\begin{itemize}
		\itemsep0em
		\item Hindi
		\item Arabisch
		\item Mandarin-Chinesisch 
		\item Spanisch
		\item Englisch
	\end{itemize}
}

%%%%%% TASK 6 %%%%%

\sortingTask{
	Sortieren Sie das Obst nach ihrer Kalorienanzahl pro 100g (1 = am wenigsten, 5 = am meisten).
}{
	\textbf{Obst:}
	\begin{itemize}
		\itemsep0em
		\item Wassermelone
		\item Rhababer
		\item Kiwi 
		\item Pfirsich
		\item Apfel
	\end{itemize}
}

%%%%%% TASK 7 %%%%%	

\sortingTask{
	Bestimmen Sie, welche der folgenden Geräte im Jahr 2017 in deutschen Haushalten am häufigsten vertreten waren (1 = am häufigsten, 5 = am seltensten).
}{
	\textbf{Haushaltsgeräte:}
	\begin{itemize}
		\itemsep0em
		\item Kaffeemaschine
		\item Mikrowellengerät
		\item Geschirrspülmaschine
		\item Gefriertruhe
		\item Waschmaschine
	\end{itemize}
}

\subsection*{\normalsize Bitte zeichnen Sie in das folgende Feld ein, wie gut Sie die bisherigen Aufgaben beantwortet haben.}
\begin{center}

\begin{tikzpicture}[scale=1]
% Variables
\def\boxHeight{5} % in cm
\def\boxWidth{15} % in cm

\def\xlabel{Prozent korrekt}

% Box
\node[rectangle, draw=black, rounded corners=0mm, minimum width=\boxWidth cm, minimum height=\boxHeight cm, line width=1pt] at (\boxWidth/2,\boxHeight/2) {};

% x-axis tiks
\def\ytop{0}
\def\ybottom{-0.1}
\draw[line width=1pt] (0, \ytop) -- (0, \ybottom) node[anchor=north] {\large $0$};
\draw[line width=1pt] (0.2*\boxWidth, \ytop) -- (0.2*\boxWidth, \ybottom) node[anchor=north] {\large $20$};
\draw[line width=1pt] (0.4*\boxWidth, \ytop) -- (0.4*\boxWidth, \ybottom) node[anchor=north] {\large $40$};
\draw[line width=1pt] (0.6*\boxWidth, \ytop) -- (0.6*\boxWidth, \ybottom) node[anchor=north] {\large $60$};
\draw[line width=1pt] (0.8*\boxWidth, \ytop) -- (0.8*\boxWidth, \ybottom) node[anchor=north] {\large $80$};
\draw[line width=1pt] (1*\boxWidth, \ytop) -- (1*\boxWidth, \ybottom) node[anchor=north] {\large $100$};

% y-axis tiks
\def\xright{0}
\def\xleft{-0.1}
\draw[line width=1pt] (\xright, 0) -- (\xleft, 0) node[anchor=east] {\large $0$};
\draw[line width=1pt] (\xright, 0.2*\boxHeight) -- (\xleft, 0.2*\boxHeight) node[anchor=east] {\large $0.2$};
\draw[line width=1pt] (\xright, 0.4*\boxHeight) -- (\xleft, 0.4*\boxHeight) node[anchor=east] {\large $0.4$};
\draw[line width=1pt] (\xright, 0.6*\boxHeight) -- (\xleft, 0.6*\boxHeight) node[anchor=east] {\large $0.6$};
\draw[line width=1pt] (\xright, 0.8*\boxHeight) -- (\xleft, 0.8*\boxHeight) node[anchor=east] {\large $0.8$};
\draw[line width=1pt] (\xright, 1*\boxHeight) -- (\xleft, 1*\boxHeight) node[anchor=east] {\large $1$};	

% Draw grid lines
% \draw[step=.5cm, color=gray] (0,0) grid (\boxWidth, \boxHeight);

% Captions
\node[] at (\boxWidth/2, -.8) {\xlabel}; % x-axis
\node[rotate=90] at (-1.1, \boxHeight/2) {Wahrscheinlichkeit}; % y-axis
\end{tikzpicture}

\end{center}

%%%%%% TASK 8 %%%%%

\sortingTask{
	Nennen und ordnen Sie die fünf europäischen Städte mit den meisten Einwohnern \\(1 = am meisten, 5 = am wenigsten).
}{	
	\textbf{Notizen:}
	
	\vspace{.5mm}
	
	\fbox{\color{white}\rule{3.5cm}{3.7cm}\color{black}}
}


\end{document}